\begin{abstract}

% What is the problem?
% * Data analysis is becoming more complicated
% * Existing systems offer

Data analytics pipelines are becoming increasingly more complicated due to the growing number of requirements imposed by data science. Not only must data be processed and analyzed scalably with respect to its volume and velocity, but also intricately by involving many different data types. Arc-Lang is a programming language for data analytics which unifies programming with streaming data and data at rest. In this paper we give a formal definition of Arc-Lang along with examples of its applications. We describe how Arc-Lang programs are compiled using the MLIR compilation framework into Rust and then deployed on a distributed system. 

% Distributed systems for data intensive computing are gaining traction as a topic of research due to the growing number of requirements imposed by data science. As implementing distributed applications is inherently hard, systems leverage \textit{domain-specific languages} (DSLs), typically oriented around high-level collection-based data-types such as tensors, tables, graphs, and streams, that allow users to focus on solving application-specific problems. Even though much work has been dedicated to empirically comparing the DSLs and their underlying systems, less progress is being made on comparing their \textit{qualitative} features. In this survey, we review and compare different high-level DSLs for data intensive computing based on their problem domains, programming models, and implementations. We identify classes of problems that illustrate the strengths and weaknesses of representative DSLs and tradeoffs associated with their implementation patterns. Finally, we discuss future trends and open research questions in the area of DSLs for data intensive computing.

\end{abstract}